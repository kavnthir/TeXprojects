 % DOC SETTINGS ===================================
\documentclass{article}
\usepackage[utf8]{inputenc}
\usepackage{steinmetz}
\usepackage{mathtools}  
\usepackage{multicol}
\usepackage{circuitikz}
\usepackage{tikz}
\usepackage{listings}
\usepackage{geometry}
\usepackage{fancyhdr}
\usepackage{amsfonts}
\usepackage{media9}
\usepackage{parskip}
\usetikzlibrary{positioning, fit, calc}
\pagestyle{fancy}
\lhead{Weekly Report 6}
\rhead{Kavin Thirukonda 2021}
\PassOptionsToPackage{hyphens}{url}\usepackage{hyperref}
\hypersetup{
    colorlinks=true,
    linkcolor=blue,
    filecolor=magenta,      
    urlcolor=cyan,
}
\fancyheadoffset{0mm}
 \geometry{
 a4paper,
 total={170mm,257mm},
 left=20mm,
 top=25mm,
 }
\usepackage{listings}
\usepackage[]{color}
\definecolor{codegreen}{rgb}{0,0.6,0}
\definecolor{codegray}{rgb}{0.5,0.5,0.5}
\definecolor{codepurple}{rgb}{0.58,0,0.82}
\definecolor{backcolour}{rgb}{0.95,0.95,0.92}
\mathtoolsset{showonlyrefs} 
\cfoot{}
% DOC SETTINGS ===================================
\begin{document}
\subsection*{EWC Power Rail Measurements}
\begin{itemize}
    \item Took DC measurements for all power rails
    \item Took DC measurements for CPU power rail with command
    \item Found typo/software thing in cpu power command
    \item started to take AC ripple measurements on power rails
    \item R\&S scope issue communication
\end{itemize}
\subsection*{EWC R1013 issue}
\begin{itemize}
    \item Took scope and DMM measurements
    \item had modification done on R1013
    \item took measurements on LEDs around R1013 and R1014/R1022
\end{itemize}
\subsection*{LabView ESS}
\begin{itemize}
    \item remastered some labview documents
    \item watched course on SCPI language so that I can begin to made a driver module for the thermal control chamber
    \item experimented with SCPI CLI to control thermal chamber for learning purposes
\end{itemize}

\end{document}