% DOC SETTINGS ===================================
\documentclass{article}
\usepackage[utf8]{inputenc}
\usepackage{fancyhdr}
\pagestyle{fancy}
\usepackage{listings}
\usepackage{geometry}
 \geometry{
 a4paper,
 total={170mm,257mm},
 left=20mm,
 top=25mm,
 }
\fancyheadoffset{0mm}
\lhead{STAT4714 Homework 12}
\rhead{Kavin Thirukonda 2021}
\usepackage{steinmetz}
\usepackage{mathtools}  
\usepackage{multicol}
\mathtoolsset{showonlyrefs} 
\cfoot{}
% DOC SETTINGS ===================================
\begin{document}
\section*{Canvas}
\subsection*{Problem 24}
To graduate this semester, you must pass Statistics 314; and you estimate that you have an 85\% chance of passing. You need to pass Math 272 or Math 444, each of which you have an 80\% chance of passing: so you will take both in hopes that you will pass at least one. Further, you must pass German 320, which you believe you have a 90\% chance of passing.Assuming your chances of passing each class you take are independent, what is your probability of graduating this semester?
\begin{equation}
    P(A) = 0.85
\end{equation}
\begin{align}
    P(B)&= P(M1||M2)\\
    &= P(M1 \& !M2) + P(M2 \& !M1) + P(M1 \& M2)\\
    &= P(M1)\cdot(1- P(M2)) + (1- P(M1))\cdot P(M2) + P(M1)\cdot P(M2)\\
    &= 0.8\cdot0.2+ 0.2\cdot0.8+ 0.8\cdot0.8 = 0.96
\end{align}
\begin{equation}
    P(C) = 0.90
\end{equation}
\begin{equation}
    P(G) = P(A) \cdot P(B) \cdot P(C) = \boxed{0.7344}
\end{equation}
\subsection*{Problem 26}
There are 3 alternative routes by which you may drive to work: Alabaster Street, Brillantine Street, and Clancy Street. It is the beginning of rush hour, and by experience Alabaster street will be closed by a car crash on average in 25 minutes, Brillantine street in 50 minutes, and Clancy Street in 40 minutes. Accident times are completely unpredictable.

If you leave for work in one hour (60 minutes), what is the probability that (at the moment you leave) a route to workwill still be open?
\begin{equation}
P(\text{A closed}) = 1 - e^{-60/25} = 0.909
\end{equation}
\begin{equation}
P(\text{B closed}) = 1 - e^{-60/50} = 0.699
\end{equation}
\begin{equation}
P(\text{C closed}) = 1 - e^{-60/40} = 0.777
\end{equation}
\begin{align}
P(\text{at least one open}) &= 1 - P(\text{all closed})\\
&=1 - P(A) * P(B) * P(C)\\
&=1 - 0.909 * 0.699 * 0.777\\
&=\boxed{0.506}
\end{align}
\newpage
\section*{Normal Probability}
\subsection*{Problem 1}
Beetlesmay be the most varied order of animals. New beetle species are authenticated, completely unpredictably, at a typical rate of one every 7.0months. A supplement to a guide is planned to be published after 20 new species have been discovered.
\subsubsection*{a)}
What are the expected value and standard deviation of the number of months (treated as a continuous measure of time) until the supplement is published?
\begin{equation}
    \sigma = \sqrt{\lambda} = \boxed{11.832}
\end{equation}
\subsubsection*{b)}
What is the probability that the supplement will be published within 10 years (120.0 months)?
\begin{equation}
    P(X < 120) = \boxed{.047}
\end{equation}
\subsubsection*{c)}
Recalculate the answer to b) using normal approximation and compare (that was certainly easier, was it not?).
\begin{align}
    P(X < 120) &= P(Z \leq -1.69)\\
    &= \boxed{.0455}
\end{align}
\subsection*{Problem 2}
At a small savings and loan company, you have to deal with your customers being victims of identity theft. This happens unpredictably to one of your customers once every 9 weeks; and no incident seems to have any connection to any other incident.
\subsubsection*{a)}
What is the probability that you will have to deal with no more than 4 incidents of identity theft in the next year (52 weeks)?
\begin{align}
P(X \leq 4) &= P(X = 0) + P(X = 1) + P(X = 2) + P(X = 3) + P(X = 4)\\
&= e^{-5.78} \cdot 5.78^0 /0! + e^{-5.78} \cdot 5.78^1 /1! + e^{-5.78} \cdot 5.78^2/2! + e^{-5.78} \cdot 5.78^3/3! + e^{-5.78} \cdot 5.78^4/4!\\
&= \boxed{0.32}    
\end{align}
\subsubsection*{b)}
When you have had 12 more incidents of identity theft, your company will send a detective to investigate. What are the expected value and variance of the time (in weeks) until a detective is sent to investigate?
\begin{equation}
    \mu = \boxed{108\text{ weeks}}
\end{equation}
\begin{equation}
    \sigma^2 = \boxed{972\text{ weeks}}
\end{equation}
\subsubsection*{c)}
Use  normal  approximation  to  find  the  probability  that  the  detective  will  be  sent  within  2 years. 
\begin{align}
    P( T < 104) &= P[Z <  (104-108) / \sigma ]\\
    &= P[Z <  -0.1283 ]\\
    &= \boxed{.449}
\end{align}
\subsection*{Problem 3}
Given that the density of a standard normal random variable Z is
\begin{equation} 
    \frac{1}{\sqrt{2\pi}}e^{-z^2/2}
\end{equation}
\subsubsection*{a)}
use integration by parts to find $E(Z^3)$. 
\begin{align}
    E(Z^3) &= \int z^3 f(z)dz\\
    &= \int z^3 \frac{1}{\sqrt{2\pi}}e^{-z^2/2}dz\\
    &= \frac{1}{\sqrt{2\pi}}\int_{-\infty}^\infty z^3e^{-z^2/2}dz\\
    &= \frac{1}{\sqrt{2\pi}}\cdot 0\\
    &= \boxed{0}\\
\end{align}
\subsubsection*{b)}
use integration by parts to find $E(Z^4)$
\begin{align}
    E(Z^4) &= \int z^4 f(z)dz\\
    &= \int z^4 \frac{1}{\sqrt{2\pi}}e^{-z^2/2}dz\\
    &= \frac{1}{\sqrt{2\pi}}\int_{-\infty}^\infty z^4e^{-z^2/2}dz\\
    &= \frac{1}{\sqrt{2\pi}}\cdot \frac{2^{\frac{5}{2}}3\sqrt{\pi}}{4}\\
    &= \boxed{3}\\
\end{align}
\end{document} 