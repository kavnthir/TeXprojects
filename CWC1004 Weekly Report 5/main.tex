 % DOC SETTINGS ===================================
\documentclass{article}
\usepackage[utf8]{inputenc}
\usepackage{steinmetz}
\usepackage{mathtools}  
\usepackage{multicol}
\usepackage{circuitikz}
\usepackage{tikz}
\usepackage{listings}
\usepackage{geometry}
\usepackage{fancyhdr}
\usepackage{amsfonts}
\usepackage{media9}
\usepackage{parskip}
\usetikzlibrary{positioning, fit, calc}
\pagestyle{fancy}
\lhead{Weekly Report 5}
\rhead{Kavin Thirukonda 2021}
\PassOptionsToPackage{hyphens}{url}\usepackage{hyperref}
\hypersetup{
    colorlinks=true,
    linkcolor=blue,
    filecolor=magenta,      
    urlcolor=cyan,
}
\fancyheadoffset{0mm}
 \geometry{
 a4paper,
 total={170mm,257mm},
 left=20mm,
 top=25mm,
 }
\usepackage{listings}
\usepackage[]{color}
\definecolor{codegreen}{rgb}{0,0.6,0}
\definecolor{codegray}{rgb}{0.5,0.5,0.5}
\definecolor{codepurple}{rgb}{0.58,0,0.82}
\definecolor{backcolour}{rgb}{0.95,0.95,0.92}
\mathtoolsset{showonlyrefs} 
\cfoot{}
% DOC SETTINGS ===================================
\begin{document}
\subsection*{Thermal Chamber Interfacing}
\begin{itemize}
    \item Located and installed proper drivers needed to communicate from labview to GPIB-USB to thermal controller to chamber.
    \item Started to learn command scheme used to control controller
    \item learned how to send commands using SCPI via GPIB to the thermal controller in labview.
    \item Research watlow thermal controller functionality and how to control it properly 
    \item figured out what drivers were needed for watlow temperature controller and installed them.
\end{itemize}
\subsection*{Data Collection}
\begin{itemize}
    \item Located and installed drivers needed for multiplexer and datalogger
    \item Did research on how to take data from thermocouples, multiplexer and datalogger and accesss that information in labview
    \item Made basic program in labview to control multiplexer functionality in labview
    \item Made basic program in labview to control datalogger functionality in labview
\end{itemize}
\subsection*{Power Supply Interfacing}
\begin{itemize}
    \item Made program to be able to control all three power supplies with one VI file, voltage, current, over-voltage and over-current protection values can independently be controlled for each power supply
    \item option available to turn all supplies on or off at the same time or to independently toggle each supply
\end{itemize}

\end{document} 