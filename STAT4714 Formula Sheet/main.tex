% DOC SETTINGS ===================================
\documentclass[landscape, 12pt]{report}
\usepackage[utf8]{inputenc}
\usepackage{fancyhdr}
\usepackage{steinmetz}
\usepackage{mathtools}
\usepackage{geometry}
\usepackage{multicol}
\usepackage{amsmath}
\geometry{
  left=0.45in, 
  right=0.45in, 
  top=0.2in, 
  bottom=.5in,
  headheight=23pt,
  includehead
}
\fancyheadoffset{0pt}
\cfoot{}
\setlength{\columnseprule}{.5pt}
\def\columnseprulecolor{}
\lhead{STAT4714 Formula Sheet}
\rhead{Kavin Thirukonda 2021}
\pagestyle{fancy}
\mathtoolsset{showonlyrefs} 
% DOC SETTINGS ===================================
\begin{document}
\begin{multicols*}{3}
\subsection*{Combinatorics}
\subsubsection*{Staging Principle}
if an outcome is determined bin k stages with $n_1$ choices at stage 1, $n_2$  choices at stage 2, and $n_k$ choices at stage k, then:
\begin{equation}
    \#\text{ total} = n_1 \cdot n_2 \cdot ... \cdot n_k
\end{equation}
\subsubsection*{Ordered List with Replacement}
Lists with a string of k things and a free choice of one of n things in each element.
\begin{equation}
    \#\text{ total} = n \cdot n \cdot ... \cdot n = n^k
\end{equation}
\subsubsection*{Ordered List without Replacement}
Count lists with k elements and n choices for each element, may not repeat a choice.
\begin{equation}
    \#\text{ lists} =P_{(n,k)} = n\cdot(n-1)\cdot...\cdot(n-k+1)
\end{equation}
\subsubsection*{Unordered Collections without Replacement}
calculate number of collections of k things taken from n.
\begin{equation}
    \#\text{ collections} =C_{(n,k)} = \frac{n!}{k!(n-k)!}
\end{equation}
\subsubsection*{Multinomial Counting}
Distribute n things in to $i = 1,2,...,k$ categories, with $n_1$ in category one, $n_2$ in category two, and $n_k$ in category k.
\begin{equation}
    \#\text{ distributions} = \frac{n!}{n_1!\cdot n_2!\cdot...\cdot  n_k!}
\end{equation}
\subsection*{Basic Probability}
\subsubsection*{Probability Laws and Notation}
\begin{center}
Event: set of possible outcomes\\
S = sample space = all outcomes\\
A$\cup$B all outcomes in A or B\\
A$\cap$B all outcomes in A and B both\\
A - B outcomes in A but not B\\
S - A all outcomes not in A\\
Law 1: P(S) = 1\\
Law 2: P(A) $\geq$ 0\\
Law 3: P(A$\cup$B) = P(A) + P(B-A)\\
If B is disjoin from A (A$\cap$B = $\phi$) then
P(A$\cup$B) = P(A) + P(B)\\
\end{center}
\subsubsection*{Conditional Probability}    
The conditional probability of event B given event A is P(B$\lvert$A)
\begin{equation}
    P(B\lvert A) = \frac{P(A\cap B)}{P(A)}
\end{equation}
\begin{equation}
    P(A\cap B) = P(A)P(B\lvert A)
\end{equation}
If B is independent of A then:
\begin{equation}
    P(A\cap B) = P(A)P(B)
\end{equation}
\subsubsection*{Division into cases}
\begin{equation}
    \sum_{i=1}^kP(B_i)P(A\lvert B_i)
\end{equation}
\columnbreak
\subsection*{Random Variables}
\subsubsection*{Binomial}
\begin{equation}
    p(x) = {n\choose x}p^x( 1-p)^{x-1}
\end{equation}
\subsubsection*{Hypergeometric}
N things in a population\\
k distinguished things\\
n randomly chosen things from the population\\
\begin{equation}
    P(\text{n has x from k}) = \frac{{k\choose x}{N-k\choose n-x}}{{N\choose n}}
\end{equation}
\subsubsection*{Geometric}
the probability of getting a specific event to  occur after x attempts if p is the probability of a singular event
\begin{equation}
    p(x) = (1-p)^{x-1}p
\end{equation}
\subsubsection*{Probability Equality Solving Technique}
If an inequality question is setup in this exact form then an answer can be easily derived using the formula.
\begin{equation}
    P(A < x \leq B) = F(B) - F(A)
\end{equation}
\end{multicols*}
\end{document}
