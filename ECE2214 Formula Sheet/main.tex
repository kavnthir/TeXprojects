% DOC SETTINGS ===================================
\documentclass[landscape, 12pt]{report}
\usepackage[utf8]{inputenc}
\usepackage{fancyhdr}
\usepackage{steinmetz}
\usepackage{mathtools}
\usepackage{geometry}
\usepackage{multicol}
\usepackage{amsmath}
\geometry{
  left=0.45in, 
  right=0.45in, 
  top=0.2in, 
  bottom=.5in,
  headheight=23pt,
  includehead
}
\fancyheadoffset{0pt}
\cfoot{}
\setlength{\columnseprule}{.5pt}
\def\columnseprulecolor{}
\lhead{ECE2214 Formula Sheet}
\rhead{Kavin Thirukonda 2021}
\pagestyle{fancy}
\mathtoolsset{showonlyrefs} 
% DOC SETTINGS ===================================
\begin{document}
\begin{multicols*}{3}
\subsubsection*{Doping}
\begin{equation}
    n_i = BT^{3/2}e^{(\frac{-E_g}{2kT})}
\end{equation}
\begin{center}
\begin{tabular}{c|c|c}
     Material & $E_g$ & B \\
     \hline
     Si& 1.1 & $5.23\cdot10^{15}$\\ 
     GaAs& 1.4 & $2.10\cdot10^{14}$\\ 
     Ge& .66 & $1.66\cdot10^{15}$\\ 
\end{tabular}   
\end{center}
\begin{equation}
     n_i^2 = np
\end{equation}
\begin{center}
    at T=300K, if $N_d >> n_i$:
\end{center}
\begin{equation}
    n_o \cong N_d
\end{equation}
\begin{center}
    at T=300K, if $N_a >> n_i$:
\end{center}
\begin{equation}
    p_o \cong N_a
\end{equation}
\subsubsection*{Excess Carriers (Non-Thermal Eq.)}
\begin{equation}
    n = n_o +\delta n
\end{equation}
\begin{equation}
    p = p_o +\delta p
\end{equation}
\begin{equation}
    n_i^2 = n_op_o
\end{equation}
\subsubsection*{Drift \& Diffusion}
\begin{equation}
    v_{dn} = -\mu_nE
\end{equation}
\begin{equation}
    v_{dn} = +\mu_pE
\end{equation}
\begin{equation}
    J_n = +en\mu_nE + eD_n\frac{dn}{dx}
\end{equation}
\begin{equation}
    J_p = +en\mu_pE -eD_p\frac{dp}{dx}
\end{equation}
\begin{equation}
    J = J_n + J_p = \sigma E = \frac{1}{\rho}E = \frac{I}{A}
\end{equation}
\begin{equation}
    R = \rho \frac{L}{A}
\end{equation}
\begin{equation}
    \sigma = en\mu_n + ep\mu_p
\end{equation}
\begin{equation}
    V_T = \frac{D_n}{\mu_n} = \frac{D_p}{\mu_p} = \frac{kT}{e} \cong 0.026V
\end{equation}
\subsubsection*{PN Junction Diode}
\begin{equation}
    V_{bi} = V_T ln (\frac{N_aN_d}{n_i^2})
\end{equation}
\begin{equation}
    C_j = C_{jo}(1+\frac{V_R}{V_{bi}})^{-1/2}
\end{equation}
\begin{equation}
    i_D = I_S [exp(\frac{v_D}{nV_T})-1]
\end{equation}
\subsubsection*{MOSFET}
\begin{equation}
    K_n = k^{'}_{n}\frac{W}{2L}
\end{equation}
\begin{equation}
    C_{ox} = \varepsilon_o/t_{ox}
\end{equation}
\begin{tabular}{p{1.5cm}|p{2.775cm}|p{2.775cm}}
Region & nMOS & pMOS \\
\hline
Lin. & $v_{DS} < v_{DS}(sat)$ & $v_{SD} < v_{SD}(sat)$ \\
Sat. & $v_{DS} > v_{DS}(sat)$ & $v_{SD} > v_{SD}(sat)$ \\
Enh. & $V_{TN} > 0V$ & $V_{TP} < 0V$ \\
Dep. & $V_{TN} < 0V$ & $V_{TP} > 0V$ \\
\end{tabular}
\subsubsection*{nMOS}
\begin{equation}
    i_D = K_n(v_{GS} - V_{TN})^2
\end{equation}
\begin{equation}
    i_D = K_n[2(v_{GS} - V_{TN})v_{DS}-v_{DS}^2]
\end{equation}
\begin{equation}
    K_n = \frac{W\mu_n C_{ox}}{2L}
\end{equation}
\begin{equation}
    v_{DS}(sat) = v_{GS} - V_{TN}
\end{equation}
\subsubsection*{pMOS}
\begin{equation}
    i_D = K_n(v_{SG} + V_{TP})^2
\end{equation}
\begin{equation}
    i_D = K_n[2(v_{SG} + V_{TP})v_{SD}-v_{SD}^2]
\end{equation}
\begin{equation}
    K_n = \frac{W\mu_p C_{ox}}{2L}
\end{equation}
\begin{equation}
    v_{SD}(sat) = v_{SG} + V_{TP}
\end{equation}
\subsubsection*{Variable Appendix}
$\boldsymbol{n_i}$ is the intrinsic carrier concentration.\\
$\boldsymbol{B}$ is the semiconductor material value. \\
$\boldsymbol{E_g}$ is the bandgap energy.\\
$\boldsymbol{k}$ is Boltzmann’s constant
$(86 \cdot 10^{-6} \frac{eV}{K})$.\\
$\boldsymbol{n_o}$ is the concentration of free electrons.\\
$\boldsymbol{p_o}$ is the concentration of holes.\\
$\boldsymbol{N_d}$ is the net donor concentrations.\\
$\boldsymbol{N_a}$ is the net acceptor concentrations.\\
$\boldsymbol{v_{dn}}$ is the drift velocity.\\
$\boldsymbol{\mu_n}$ is the electron mobility constant\\ typically  $(1350\frac{cm^2}{V\cdot s})$.\\
$\boldsymbol{v_{dp}}$ is the drift velocity.\\
$\boldsymbol{\mu_p}$ is the hole mobility constant\\ typically $(480\frac{cm^2}{V\cdot s})$.\\
$\boldsymbol{E}$ is an electric field.\\
$\boldsymbol{J_n}$ is the electron current density.\\
$\boldsymbol{J_p}$ is the hole current density.\\
$\boldsymbol{e}$ is the electric charge value $(1.6\cdot10^{-19})$.\\
$\boldsymbol{\sigma}$ is the conductivity.\\
$\boldsymbol{\rho}$ is the resistivity.\\
$\boldsymbol{A}$ is the cross-sectional area.\\
$\boldsymbol{D_n}$ is the electron diffusion coefficient.\\
$\boldsymbol{D_p}$ is the hole diffusion coefficient.\\
$\boldsymbol{V_T}$ is the thermal voltage.\\
$\boldsymbol{\delta n }$ is the excess electron concentration.\\
$\boldsymbol{\delta p }$ is the excess hole concentration.\\
$\boldsymbol{C_j}$ is the junction capacitance.\\
$\boldsymbol{C_{jo}}$ is the junction capacitance w/ no voltage.\\
$\boldsymbol{V_{R}}$ is the voltage applied across the junction.\\
$\boldsymbol{I_{S}}$ is the reverse-bias saturation current.\\
$\boldsymbol{n}$ is the emission coefficient (1 unless stated).\\
$\boldsymbol{K_n}$ is the conduction parameter.\\
$\boldsymbol{C_{ox}}$ is the oxide capacitance.\\
\end{multicols*}
\begin{multicols*}{3}
\subsubsection*{MOSFET Small-Signal}
\begin{center}
\begin{tabular}{c|c}
     Notation & Meaning \\
     \hline
     $i_D, v_{GS}$& Total instantaneous values  \\
     $I_D, V_{GS}$& DC Values \\
     $i_d, v_{gs}$& Instantaneous AC values \\
     $I_d, V_{gs}$& Phasor Values \\
\end{tabular}  
\begin{equation}
    g_m = \frac{i_d}{v_{gs}} = 2\sqrt{K_nI_{DQ}}
\end{equation}
\begin{equation}
    r_o \cong \frac{1}{\lambda I_{DQ}} = \frac{V_A}{I_{DQ}} = \frac{\partial i_D}{\partial v_{DS}}^{-1}\bigg|_{v_{GS}=V_{GSQ}=const}
\end{equation}
\begin{equation}
    A_v = \frac{V_o}{V_i} 
\end{equation}
\subsubsection*{nMOS}
common source
\begin{equation}
    A_v  = -g_m(r_o||R_D)(\frac{R_i}{R_i+R_{Si}})
\end{equation}
common drain\\
common gate
\subsubsection*{pMOS}
common source\\
common drain\\
common gate
\end{center}
\subsubsection*{Resistive Load Inverter}
\subsubsection*{Enhancement Load Inverter}
\subsubsection*{CMOS Inverter}
$\boldsymbol{g_m}$ is the transconductance.\\
$\boldsymbol{r_O}$ is the variable resistance.\\
$\boldsymbol{\lambda}$ is the channel-length modulator.\\
$\boldsymbol{A_v}$ is the small-signal voltage gain.
\end{multicols*}
\end{document}
