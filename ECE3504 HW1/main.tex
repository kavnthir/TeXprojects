% DOC SETTINGS ===================================
\documentclass{article}
\usepackage[utf8]{inputenc}
\usepackage{fancyhdr}
\pagestyle{fancy}
\usepackage{geometry}
 \geometry{
 a4paper,
 total={170mm,257mm},
 left=20mm,
 top=25mm,
 }
\fancyheadoffset{0mm}
\lhead{ECE3504 HW1}
\rhead{Kavin Thirukonda 2021}
\usepackage{steinmetz}
\usepackage{listings}
\usepackage{circuitikz}
\usepackage{mathtools}  
\mathtoolsset{showonlyrefs} 
\cfoot{}
% DOC SETTINGS ===================================
\begin{document}
\section*{Problem 1 (10 points)}
There are many different examples of Instruction Set Architectures (ISAs).Give three examples of ISAs other than MIPS. For each example, give the following details:
\begin{itemize}
    \item What differentiates the ISA from MIPS?
    \item What similarities does the ISA share with MIPS?
    \item What devices or applications most often use the ISA?
\end{itemize}
\begin{enumerate}
        \item x86-64
        \begin{enumerate}
            \item 64-bit registers
            \item Can use 32-bit addresses
            \item Desktop Computers, game consoles
        \end{enumerate}
        \item ARM
        \begin{enumerate}
            \item More addressing modes
            \item Similar core of instructions
            \item Smartphones, Laptops
        \end{enumerate}
        \item IA-64
        \begin{enumerate}
            \item More general purpose registers
            \item Uses registers for operations
            \item High performance computers
        \end{enumerate}
    \end{enumerate}
\section*{Problem 2 (20 points)}
Assume that the base address 0x1000 of the array A is stored in \$s0.  The following table gives the memory addresses of A[0], ..., A[3] along with the values stored in the corresponding memory location.

\begin{center}
    \begin{tabular}{c|c|c}
         & Stored Value &  Memory Address\\
         \hline
         A[0] & 8 & 0x1000\\
         A[1] & -2 & 0x1004 \\
         A[2] & -3 & 0x1008 \\
         A[3] &  10 & 0x100C 
    \end{tabular}
\end{center}

Show the values in the array and give the values stored in registers \$s0, \$s1, \$s2, and \$s3 after executing the following MIPS instructions.
\vspace{5mm}

lw   \$s1 ,   0 ( \$ s 0 )

sw   \$s1 ,   4 ( \$ s 1 )

lw   \$s2 ,   8 ( \$ s 0 )

add   \$s3 ,   \$s1 ,   \$ s 2

\vspace{5mm}

lw   \$s1 ,   0 ( \$ s 0 ) $\rightarrow$ loads word at 0x0000 into \$s1

sw   \$s1 ,   4 ( \$ s 1 ) $\rightarrow$ stores word at \$s1 in a byte address which changes A[3] to 8

lw   \$s2 ,   8 ( \$ s 0 ) $\rightarrow$ loads word at 0x0000 + 8 into \$s2

add  \$s3 , \$s1 , \$ s 2 $\rightarrow$ adds \$s1 and \$s2 and stores it into \$s3

\begin{center}
    A[0] = 8, A[1] = -2, A[2] = -3, A[3] = 8, \$s0 = 8, \$s1 = 8, \$s2 = -3, \$s3 = 5
\end{center}


\newpage
\section*{Problem 3 (70 points)}
Convert each of the following C statements into MIPS instructions.  Assume that the variables a, b,and c are stored in \$s0, \$s1, and \$s2, respectively.  The base addresses of the arrays A and B are stored in \$s3 and \$s4, respectively.  Assume the arrays A and B contain integers (4 bytes).
\begin{enumerate}
    \item a = b - A[$2^{15} + 1$]
    \begin{enumerate}
        \item lw \$t0, 131076(\$s3) 
        \item sub \$s0, \$s1, \$t0
    \end{enumerate}
    \item B[3] = c + 4*a
    \begin{enumerate}
        \item sll \$t0, \$s0, 2
        \item add \$t1, \$s2, \$t0
        \item sw \$t1, 12(\$s4)
    \end{enumerate}
    \item A[7] = 2*B[3] - b + A[c]
    \begin{enumerate}
        \item sll \$t0, \$s2, 2
        \item add \$t1, \$s3, \$t0
        \item lw \$t2, 12(\$s4)
        \item sll \$t3, \$t2, 1
        \item sub \$t4, \$t3, \$s1
        \item add \$t5, \$t4, \$t1
        \item sw \$t5, 28(\$s3)
    \end{enumerate}
    \item A[B[2]] = B[a + $2^{10}$] + c
    \begin{enumerate}
        \item addi \$t0, \$s0, 0x400
        \item add \$t1, \$t0, \$s4
        \item add \$t2, \$t1, \$s2
        \item lw \$t3, 4(\$s4)
        \item add \$t4, \$t3, \$s3
        \item sw \$t2, \$t4
    \end{enumerate}
\end{enumerate}

\end{document}
