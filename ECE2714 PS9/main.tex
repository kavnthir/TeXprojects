% DOC SETTINGS ===================================
\documentclass{article}
\usepackage[utf8]{inputenc}
\usepackage{steinmetz}
\usepackage{mathtools}  
\usepackage{multicol}
\usepackage{circuitikz}
\usepackage{tikz}
\usepackage{listings}
\usepackage{geometry}
\usepackage{fancyhdr}
\usepackage{amsfonts}
\usepackage{media9}
\usepackage{parskip}
\usetikzlibrary{positioning, fit, calc}
\pagestyle{fancy}
\lhead{ECE2714 Problem Set 9}
\rhead{Kavin Thirukonda 2021}
\fancyheadoffset{0mm}
 \geometry{
 a4paper,
 total={170mm,257mm},
 left=20mm,
 top=25mm,
 }
\mathtoolsset{showonlyrefs} 
\cfoot{}
% DOC SETTINGS ===================================
\begin{document}
\begin{enumerate}
    \item Consider the time-domain signal
    \begin{equation}
        x(t) = \begin{cases}
        e^{5t} &  t < 0\\
        e^{-t} & t \geq 0
        \end{cases}
    \end{equation}
    \begin{enumerate}
        \item Find the Fourier transform of x(t) using the definition
        \begin{align}
            X(j\omega) &= \int_{-\infty}^\infty x(t) e^{-j\omega t}dt\\
            &= \int_{-\infty}^0 e^{-t} e^{-j\omega t}dt+\int_{1}^\infty e^{5t} e^{-j\omega t}dt\\
            &= \frac{1}{-j\omega - 1}e^{-j\omega t-t}\Bigg|_{-\infty}^0 + \frac{1}{-j\omega +5}e^{-j\omega t+5t}\Bigg|_{1}^\infty\\
        \end{align}
        \item Plot the Magnitude and Phase Spectrum of the Signal from -20 to 20 radians per seconds
    \end{enumerate}
    \newpage
    \item Consider the signal represented in the freqency domain as
    \begin{equation}
        X(j\omega) = \frac{19+j6\omega}{180-\omega^2+j11\omega}
    \end{equation}
    \begin{enumerate}
        \item Write the signal in the form
        \begin{equation}
            X(j\omega) = \frac{A}{2+j\omega}+\frac{B}{9+j\omega}
        \end{equation}
        i.e. determine the constants A and B
        \item Using your result from part a) find the signal representation in the time-domain i.e. x(t)
    \end{enumerate}
    \newpage
    \item Consider the continuous-time rectangular pulse signal, x(t), given by the expression,
    \begin{equation}
        x(t) = \begin{cases}
        1, & -a < t < a \\
        0, & else
        \end{cases}
    \end{equation}
    Determine the continuous-time Fourier transform $X(\omega)$ from the definition.
    \begin{align}
        X(j\omega) &= \int_{-\infty}^\infty x(t) e^{-j\omega t}dt\\
        &= \int_{-a}^a e^{-j\omega t}dt\\
        &= \frac{1}{j\omega} e^{-j\omega t}\bigg|_{-a}^a\\
        &= \frac{1}{j\omega} e^{-j\omega a}-e^{j\omega a}\\
        &= -\frac{1}{j\omega} (e^{j\omega a}-e^{-j\omega a})\\
    \end{align}
    \newpage
    \item Consider the discrete-time rectangular pulse signal, x[n], given by the expression,
    \begin{equation}
        x[n] = u[n] -u[n-N].
    \end{equation}
    Determine the discrete-time Fourier transform $X(e^{j\omega}$ from the definition.
    \item Consider the  continuous time signal, y(t), given by the expression,
    \begin{equation}
        y(t) = x(t)cos(\omega_ot)
    \end{equation}
    where
    \begin{equation}
        x(t) = \begin{cases}
        1 & |t| < T\\
        0 & else
        \end{cases}
    \end{equation}
    Determine the Fourier Transform, $Y(j\omega)$ of y(t) using the multiplication property of the Fourier Transform.
    \item Consider the discrete time signal, x[n], given by the expresion,
    \begin{equation}
        x[n] a^nu[n] 
    \end{equation}
    where $|a| < 1$.
    \begin{enumerate}
        \item Determine the Discrete Time Fourier Transform, $X(e^{j\omega})$ of x[n] using the definition of the Discrete Time Fourier Transform.
        \item Create an expression for $|X(e^{j\omega})|$ (magnitude spectrum). Simplify as much as possible.
        \item Create an expression for \phase{X(e^{j\omega})} (phase spectrum). Simplify as much as possible.
    \end{enumerate}
\end{enumerate}
\end{document}
