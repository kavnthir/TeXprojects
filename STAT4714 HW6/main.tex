% DOC SETTINGS ===================================
\documentclass{article}
\usepackage[utf8]{inputenc}
\usepackage{fancyhdr}
\pagestyle{fancy}
\usepackage{listings}
\usepackage{geometry}
 \geometry{
 a4paper,
 total={170mm,257mm},
 left=20mm,
 top=25mm,
 }
\fancyheadoffset{0mm}
\lhead{STAT4714 Homework 6}
\rhead{Kavin Thirukonda 2021}
\usepackage{steinmetz}
\usepackage{mathtools}  
\usepackage{multicol}
\mathtoolsset{showonlyrefs} 
\cfoot{}
% DOC SETTINGS ===================================
\begin{document}
\section*{Canvas}
\subsection*{Problem 15}
Approximately 26\% of people have Rh negative blood. We will check the blood tests of the next 12 people to enter an emergency clinic with injuries that may require transfusion. Assume no two belong to the same family.
\subsubsection*{a)}
What is the probability that exactly 5 of those patients will have Rh negative blood?
\begin{equation}
    P(X=5 |12,0.26) = C_{(12,5)}*(0.26^5)*(1-0.26)^{(12-5)} = 792*0.001188*0.1215 = 0.1143
\end{equation}
\subsubsection*{b)}
What is the probability that at least 3 of them will have Rh negative blood?
\begin{align}
    P(X>=3 |12,0.26) &= 1 - P(X<3)\\
    &= 1 - P(X=0) - P(X=1) - P(X=2)\\ 
    &= 1 - C_{(12,0)}(.74)^{(12)} -                    C_{(12,1)}(0.26)(.74)^{(11)} -                    C_{(12,2)}(0.26^2)(.74)^{(10)}\\
    &= 1 - 0.027 - 0.1137 - 0.2197\\ 
    &= 1 - 0.3603\\ 
    &= 0.6397
\end{align}
\subsubsection*{c)}
What are the expected number and standard deviation of the number of these patients with Rh negative blood?
\begin{equation}
    E(X) = n*p = 12 * 0.26 = 3.12
\end{equation}
\begin{equation}
    \sigma = \sqrt{npq} = \sqrt{12*0.26*(1-0.26)} = 1.52
\end{equation}
\subsection*{Problem 16}
Historically 80\% of the monitors made by your company will pass your stringent quality control checks. 40 monitors have just come off the assembly line.
\subsubsection*{a)}
What is the probability that exactly 30 of them will pass the quality control checks?
\begin{equation}
    P(X = 30) = C_{(40,30)}*.80^{30}*.20^{10} = .1075
\end{equation}
\subsubsection*{b)}
What are the expected value and variance of the number of these monitors that will pass the quality control checks?
\begin{equation}
    \mu = np = 40*.80 = 32
\end{equation}
\begin{equation}
    \sigma^2 = \mu q = 32 * .20 = 6.4
\end{equation}
\newpage
\subsection*{Problem 17}
Inventing is a difficult way to make money. Only 5\% of new patents earn a substantial profit. A certain city has just had 30 independent new patents recorded.
\subsubsection*{a)}
What is the probability that at least 2 of these new patents will earn a substantial profit?
\begin{align}
    P(X\geq2) &= 1- P(X<=1) \\
    &= 1-P(X=0)-P(X=1) \\
    &= 1-C_{(30,0)}(0.95)^{30} -C_{(30,1)}(0.05)(0.95)^{29}\\
    &=0.4465 
\end{align}
\subsubsection*{b)}
Justify an appropriate simplified approximate method for answering the question in a), carry it out, and compare the two answers.
\begin{align}
P(X>=2) &= 1-P(X<=1) \\
&=1-P(X=0)-P(X=1) \\
&=1-e-\frac{1.5*1.5^0}{0!}-e-\frac{1.5*1.5^1}{1!}\\
&= .4422
\end{align}
\subsection*{Problem 23}
A  small  press  publishes  an  unpredictable  number X of new books  each  month  with probability generating function
\begin{equation}
    \pi_X(q) = E(q^X)=\frac{(4+5q)(3+7q)}{100}
\end{equation}
\subsubsection*{a)}
Each book they publish in a given month has an 80\% chance of reaching a best-seller list. What is the probability that next month all the books they publish will reach the best-seller list? 
\begin{align}
    P &= E[(.8)^x]\\
    &= \frac{(3+6*.8)(3+7*.8)}{100}\\
    &= \frac{(8.8)(8.6)}{100}\\
    &= .7568
\end{align}
\subsubsection*{b)}
Calculate the expected value and variance of the number of books they will publish next month.
\begin{equation}
    E[x] = \frac{84+46}{100} = 1.3
\end{equation}
\begin{equation}
    E[x^2] = \frac{84}{100} + 1.3 = 2.14
\end{equation}
\begin{equation}
    \sigma^2 = 2.14 - 1.3^2  = .45
\end{equation}
\newpage
\subsection*{Problem 24}
The number of houses a trick-or-treating child will get candy at before some house gives them a “healthy” snack follows a random variable H= {0, 1, 2, 3, ...} with probability generating function
\begin{equation}
    \pi_H(q)=E(q^H)=\frac{1}{5-4q}
\end{equation}
\subsubsection*{a)}
Calculate the expected value and variance of the number of houses H at which a child will get candy.
\begin{equation}
    E(H) = \frac{\frac{4}{5}}{\frac{1}{5}}=4
\end{equation}
\begin{equation}
    \sigma^2 = \frac{\frac{4}{5}}{\frac{1}{5}^2} = 20
\end{equation}
\subsubsection*{b)}
The houses at which a child gets candy each have a probability of .75 of including chocolate. What is the probability that all H houses give a child chocolate?
\begin{equation}
    P(B) = .75^H
\end{equation}
\end{document} 