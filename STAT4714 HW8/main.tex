% DOC SETTINGS ===================================
\documentclass{article}
\usepackage[utf8]{inputenc}
\usepackage{fancyhdr}
\pagestyle{fancy}
\usepackage{listings}
\usepackage{geometry}
 \geometry{
 a4paper,
 total={170mm,257mm},
 left=20mm,
 top=25mm,
 }
\fancyheadoffset{0mm}
\lhead{STAT4714 Homework 8}
\rhead{Kavin Thirukonda 2021}
\usepackage{steinmetz}
\usepackage{mathtools}  
\usepackage{multicol}
\mathtoolsset{showonlyrefs} 
\cfoot{}
% DOC SETTINGS ===================================
\begin{document}
\section*{Canvas}
\subsection*{Problem 1}
On the interstate, you have just passed a franchise of your favorite fast-food chain, and you happen to know that there is another one in 40 miles; but you are very hungry and want to stop earlier. You are pretty sure that there is exactly one restaurant located closer; and the reliability function for the number of miles Muntil you reach it is $R(x) = P(X>x) = 1 - x^2/1600$.
\subsubsection*{a)}
On average, how many miles will you have to drive to find this restaurant?
\begin{align}
    E(x) &= \int_0^{40} x \cdot F(x)dx\\
    &= \frac{1}{800}\left(\frac{x^3}{3}\right)^{40}_0\\
    &= \frac{1}{800}\left(\frac{64000}{3}\right)\\
    &= \boxed{26.66\text{ miles}}
\end{align} 
\subsubsection*{b)}
What is the probability you will have driven 30 miles, and still not found the restaurant?
\begin{equation}
    R(30) = 1 - \frac{30^2}{1600} = \boxed{.4375}
\end{equation}
\subsection*{Problem 2}
The consumer taste evaluation of a certain microwave dinner (on a continuous scale 0-10) X has cumulative distribution function $F(x) = P(X\leq x) = \frac{x}{120}(2+x)$.
\subsubsection*{a)}
Find the probability that the next consumer will evaluate the dinner as higher than a 5 but no higher than an 8.
\begin{equation}
    P(A) = P(5<X<8)=F(8)-F(5) =(8/120)*(2+8) =(5/120)*(2+5) = \boxed{0.375}
\end{equation}
\subsubsection*{b)}
Calculate the expected value and variance of consumer evaluations.
\begin{equation}
    var(x) = E(X^2) - E(X)^2 = \int_0^{10}x\cdot F(x)dx - \int_0^{10}x^2\cdot F(x)dx = \boxed{6.40}
\end{equation}
\newpage
\subsection*{Problem 4}
The working lifetime of a certain brand of flashlight battery $0 < T< 12$, where T is in months, follows a probability density f(t) = (12 –t)/72.
\subsubsection*{a)}
Find the expected value and standard deviation of the working lifetime of this battery.
\begin{align}
    E(T) &= \int_0^{12}tf(t)dt\\
    &= \int_0^{12}\frac{12t-t^2}{72}dt\\
    &= \frac{1}{72}\left(\int_0^{12}12t-\int_0^{12}t^2dt\right)\\
    &= \boxed{4\text{ months}}
\end{align}
\begin{align}
    E(T^2) &= \int_0^{12}t^2f(t)dt\\
    &= \int_0^{12}\frac{12t-t^3}{72}dt\\
    &= \frac{1}{72}\left(\int_0^{12}12t-\int_0^{12}t^3dt\right)\\
    &= \boxed{24\text{ months}}
\end{align}
\begin{equation}
    var(x) = 24 - 4^2 = \boxed{8}
\end{equation}
\begin{equation}
    \sigma = \sqrt{\sigma^2} = \boxed{2.83}
\end{equation}
\subsubsection*{b)}
What is the probability that a battery of this brand will still work after 9 months?
\begin{align}
    P(T > 9) &= \int_9^{12} f(t)dt\\
    &=\int_9^{12}\frac{(12-t)}{72}dt\\
    &= \boxed{0.0625}
\end{align}
\newpage
\subsection*{Problem 5}
Police dogs are automatically retired after 5 years of service, but the dogs who retire early because of injuries do so at times T that follow a (continuous) density function $f(t) = 2t/25 0 < t< 5$ where t is in years.
\subsubsection*{a)}
Find the expected value and standard deviation of the service life of dogs who retire early.
\begin{align}
    E(t) &= \int_0^5tf(t)dt\\
    &=int_0^5\frac{2t^2}{25}dt\\
    &= \boxed{\frac{10}{3}}
\end{align}
\begin{align}
    E(t^2) &= int_0^5 t^2f(t)dt\\
    &= \int_0^5 \frac{2t^3}{25}\\
    &= \boxed{\frac{25}{2}}
\end{align}
\begin{equation}
    var(x) = E(T^2) - E(T)^2 = \boxed{\frac{25}{18}}
\end{equation}
\subsubsection*{b)}
What is the probability that a dog who retires early will have at least 4 years of service?
\begin{equation}
    P(t\geq4) = \int_4^5 f(t) dt = \boxed{\frac{9}{25}}
\end{equation}
\subsection*{Problem 8}
At a certain always-busy intersection, a major car crash happens every 13.5 days on average.
\subsubsection*{a)}
What is the probability the next accident will happen within 10.0 days?
\begin{equation}
    P(T\leq10) = 1-e^{-10/13.5} = \boxed{.5232}
\end{equation}
\subsubsection*{b)}
What is the variance of the time until that next accident happens?
\begin{equation}
    var(x) = \frac{1}{\lambda^2} = \boxed{182.25}
\end{equation}
\end{document} 