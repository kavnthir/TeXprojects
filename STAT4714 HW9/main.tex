% DOC SETTINGS ===================================
\documentclass{article}
\usepackage[utf8]{inputenc}
\usepackage{fancyhdr}
\pagestyle{fancy}
\usepackage{listings}
\usepackage{geometry}
 \geometry{
 a4paper,
 total={170mm,257mm},
 left=20mm,
 top=25mm,
 }
\fancyheadoffset{0mm}
\lhead{STAT4714 Homework 9}
\rhead{Kavin Thirukonda 2021}
\usepackage{steinmetz}
\usepackage{mathtools}  
\usepackage{multicol}
\mathtoolsset{showonlyrefs} 
\cfoot{}
% DOC SETTINGS ===================================
\begin{document}
\section*{Canvas}
\subsection*{Problem 3}
The time in days from the beginning of each year when the first bud appears on a pumpkin plant follows a continuous random variable $50< D<100$ whose cumulative distribution function is
\begin{equation}
    F(d) = P(D \leq d) = \frac{(d - 50)^2}{2500}
\end{equation}
\subsubsection*{a)}
What is the probability that the plant will bud within 80.0 days of the beginning of the year?
\begin{equation}
    \frac{(80-50)^2}{2500} = \boxed{.36}
\end{equation}
\subsubsection*{b)}
What are the expected value and standard deviation of the time in days when the first bud appears?
\begin{equation}
    E(D) = \int_{50}^{100} \frac{2d(d-50)}{2500} = 83.33
\end{equation}
\begin{equation}
    E(D^2) = \int_{50}^{100} \frac{2d^2(d-50)}{2500} = 7083.33
\end{equation}
\begin{equation}
    var(x) = E(D^2) - E(D) = \boxed{139.44}
\end{equation}
\subsection*{Problem 9}
Two independent, completely accidental scientific discoveries will be required to plan a space elevator. The first, Negative Gravity, we believe will take an average of 10 years to happen. The second, Monofilament Cables, we believe will take an average of 6 years to take place.
\subsubsection*{a)}
What is the probability Monofilament Cables will be discovered within the next 10 years?
\begin{equation}
    P(Y < 10) = 1 - e^{(-10/6)} = \boxed{0.8111}
\end{equation}
\subsubsection*{b)}
What is the probability both discoveries will take place within the next 20 years?
\begin{equation}
    P(X < 20)P(Y < 20) = (1 - e^{-20/10})(1 - e^{-20/6}) = \boxed{0.8338}
\end{equation}
\subsubsection*{c)}
What is the expected time until we can plan a space elevator, because both discoveries will have taken place?
\begin{equation}
    E(Z) = 6 + 10 - 1/(1/10 + 1/6) = \boxed{12.25}
\end{equation}
\newpage
\subsection*{Problem 10} 
Extensive analysis of the time (in years) until major breakdown of a popular brand of motorcycle shows that it follows a hazard function $\lambda(t) = .4 + .3t$
\subsubsection*{a)}
Is this an aging system, a burn-in system, a constant-hazard system, or none of the above? Justify your answer.
\begin{center}
    This problem cannot be a burn in system because there is no sudden breakdown instead the system breaks down only with time
\end{center}
\subsubsection*{b)}
Remembering that the hazard function can be interpreted as a quick way to estimate how likely short-term failure is, $p= \lambda(t)\delta t$, calculate the probability that a motor cycle that has run flawlessly for 15 months will have a major breakdown in the next week
\begin{equation}
    (.4 + .3(1.25))\frac{1}{52} = \boxed{.0149}
\end{equation}
\subsubsection*{c)}
This brand of motorcycle has a 2-year warranty. Find the probability that a new 'cycle will have a major breakdown before the warranty expires.
\begin{equation}
    P = 1 - 0.2466 = \boxed{0.7534}
\end{equation}
\subsection*{Problem 11}
A new monitor has lifetime T (in years) with reliability function $R(t) = P( T > t) = \frac{4}{(1+t)^2}$ where $t > 1$
\subsubsection*{a)}
What is the probability it will have a breakdown within 3.0 years?
\begin{equation}
    P(T<3)=1-P(T>3)=1-4/(1+3)2 =1-4/16 = \boxed{3/4}
\end{equation}
\subsubsection*{b)}
Calculate and simplify the hazard function $\lambda(t) = f(t)/R(t)$. Is this an aging, burn-in, or constant hazard system? Justify your answer.
\begin{equation}
    \boxed{\frac{2}{1+t}}
\end{equation}
\begin{center}
    $\lambda(t)$ decreases with increasing time; it is burn in system
\end{center}
\newpage
\subsection*{Problem 12}
Recall that a Weibull law for the lifetime of a system is most simply described by its cumulative hazard function
\begin{equation}
    H(t) = \int_0^t \lambda(T)dT = (\frac{t}{b})^a, \text{where a, b, t} >0
\end{equation}
The heavy cables that support a bridge are theorized to each have time to breakage that follows a Weibull law with a= 2.5 and b= 30 years.
\subsubsection*{a)}
Find the hazard function $\lambda(t)$ for one of these cables. Is it what we have called an aging system, or a constant-hazard system, or a burn-in system? Explain.
\begin{center}
    Aging System, because a is greater than one.
\end{center}
\subsubsection*{b)}
Use the results of a), and our result that the probability of short-term failure of a working system is approximately $\lambda(t)\delta t$, to find roughly the probability that a 40-year old intact cable will fail in the next month.
\begin{equation}
    P = \lambda(40)*(\frac{1}{12}) = \boxed{.01069}
\end{equation}
\subsubsection*{c)}
Find the probability that a cable will last at least 50 years.
\begin{equation}
    P(t >= 50) = 1 - e^{-\frac{50}{30}^{2.5}} = \boxed{.9723}
\end{equation}
\end{document} 