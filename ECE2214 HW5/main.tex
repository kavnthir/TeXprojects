% DOC SETTINGS ===================================
\documentclass{article}
\usepackage[utf8]{inputenc}
\usepackage{steinmetz}
\usepackage{mathtools}  
\usepackage{multicol}
\usepackage{circuitikz}
\usepackage{listings}
\usepackage{geometry}
\usepackage{fancyhdr}
\usepackage{amssymb}
\usepackage{pifont}
\usetikzlibrary{positioning, fit, calc}
\pagestyle{fancy}
\lhead{ECE2214 Homework 5}
\rhead{Kavin Thirukonda 2021}
\fancyheadoffset{0mm}
\geometry{
 a4paper,
 total={170mm,257mm},
 left=20mm,
 top=25mm,
 }
\mathtoolsset{showonlyrefs} 
\cfoot{}
% DOC SETTINGS ===================================
\begin{document}
\begin{enumerate}
    \item (15 Points) Using the differential length dl, find the length of each of the following curves:
    \begin{enumerate}
        \item $\rho = 3, \pi/4 < \phi < \pi/2$, z = constant
        \begin{align}
            &\int_{\frac{\pi}{4}}^{\frac{\pi}{2}}3d\phi\\
            &=3\phi\bigg|_{\frac{\pi}{4}}^{\frac{\pi}{2}}\\
            &=3(\frac{\pi}{2}) - 3(\frac{\pi}{4})\\
            &=\frac{6\pi}{4} - \frac{3\pi}{4}\\
            &=\boxed{\frac{3\pi}{4}}
            \end{align}
        \item $r = 1, \theta = 30^\circ, 0 < \phi < 60^\circ$
        \begin{align} 
            &\int_{0^\circ}^{60^\circ}\sin(30^\circ)d\theta\\
            &=\frac{\theta}{2}\bigg|_{0^\circ}^{60^\circ}\\
            &=\boxed{\frac{\pi}{6}}
        \end{align}
        \item $r = 4, 30^\circ < \theta < 90^\circ, \phi$ = constant
        \begin{align}
            &\int_{30^\circ}^{90^\circ}4d\theta\\
            &=4\theta\bigg|_{30^\circ}^{90^\circ}\\
            &=4(90^\circ) - 4(30^\circ)\\
            &=\frac{6\pi}{3}  - \frac{2\pi}{3}\\
            &=\boxed{\frac{4\pi}{3}}
        \end{align}
    \end{enumerate}
    \newpage
    \item (20 Points) Calculate the area of the following surfaces using the differential surface area dS:
    \begin{enumerate}
        \item $\rho = 2, 0 < z <5, \pi/3 < \phi < \pi/2$
        \begin{align}
            &\int_0^5\int_{\frac{\pi}{3}}^{\frac{\pi}{2}}2d\phi dz\\
            &=\int_0^5\frac{\pi}{3}dz\\
            &=\boxed{\frac{5\pi}{3}} 
        \end{align}
        \item $z = 1, 1 < \rho < 3, 0 < \phi < \pi/4$
        \begin{align}
            &\int_0^{\frac{\pi}{4}}\int_1^3\rho d\rho d\phi\\
            &=\int_0^{\frac{\pi}{4}}\frac{1}{2}(3^2-1^2)d\phi\\
            &=\int_0^{\frac{\pi}{4}}4 d\phi\\
            &= \boxed{\pi}
        \end{align}                                                         
        \item $r = 10, \pi/4 < \theta < 2\pi/3, 0 < \phi < 2\pi$
        \begin{align}
            &\int_0^{2\pi}\int_\frac{\pi}{4}^{\frac{2\pi}{3}}r^2\sin(\theta)d\theta d\phi\\
            &=\int_0^{2\pi}\int_\frac{\pi}{4}^{\frac{2\pi}{3}}r^2\sin(\theta)d\theta d\phi\\
            &=\int_0^{2\pi}-r^2\cos(\theta)\bigg|_\frac{\pi}{4}^{\frac{2\pi}{3}}d\phi\\
            &=\int_0^{2\pi}-r^2(\cos(\frac{2\pi}{3})-\cos(\frac{\pi}{4}))d\phi\\
            &=\int_0^{2\pi}-100\left(\frac{-\sqrt{2}-1}{2}\right)d\phi\\
            &=200\pi\left(\frac{\sqrt{2}+1}{2}\right)\\
            &= \boxed{100\pi(\sqrt{2} + 1)}
        \end{align}
        \item $0 < r < 4, 60^\circ < \theta < 90^\circ, \phi =  constant$
        \begin{align}
            &\int_{60^\circ}^{90^\circ}\int_0^4 rdrd\theta\\
            &=\int_{60^\circ}^{90^\circ} 8d\theta\\
            &=8\cdot30^\circ\\
            &=\boxed{240^\circ}
        \end{align}
    \end{enumerate}
    \newpage
    \item (15 Points) Use the differential volume dv to determine the volumes of the following regions:
    \begin{enumerate}
        \item $0 < x < 1, 1 < y < 2, -3 < z < 3$
        \begin{align}
            &\int_{-3}^3\int_1^2\int_0^1dxdydz\\
            &=\int_{-3}^3\int_1^2dydz\\
            &=\int_{-3}^3dz\\
            &=\boxed{6}
        \end{align}
        \item $2 < \rho < 5, \pi/3 < \phi < \pi, -1 < z < 4$
        \begin{align}
            &\int_{-1}^4\int_\frac{\pi}{3}^\pi\int_2^5 \rho d\rho d\phi dz\\
            &=\int_{-1}^4\int_\frac{\pi}{3}^\pi\frac{1}{2}(25-4) d\phi dz\\
            &=\int_{-1}^4\frac{21}{2}(\pi-\frac{\pi}{3})dz\\
            &=\frac{42\pi}{6}(4+1)\\
            &=\boxed{35\pi}
        \end{align}
        \item $1 < r < 3, \pi/2 < \theta < 2\pi/3,\pi/6 < \phi < \pi/2$
        \begin{align}
            &\int_\frac{\pi}{6}^\frac{\pi}{2}\int_\frac{\pi}{2}^\frac{2\pi}{3}\int_1^3 r^2\sin(\theta)drd\theta d\phi\\
            &=\int_\frac{\pi}{6}^\frac{\pi}{2}\int_\frac{\pi}{2}^\frac{2\pi}{3} \frac{1}{3}(3^3-1^3)\sin(\theta)d\theta d\phi\\
            &=\int_\frac{\pi}{6}^\frac{\pi}{2} \frac{26}{3}(\cos(\frac{2\pi}{3})-\cos(\frac{\pi}{2}))d\phi\\
            &=-\frac{26}{6}(\frac{\pi}{6}-\frac{\pi}{2})\\
            &= \boxed{\frac{13\pi}{9}}
        \end{align}
    \end{enumerate}
    \newpage
    \item (10 Points) Let $H = 4\rho^2a_p - 2za_z.$ Verify the divergent theorem for the cylindrical region defined by $\rho = 10, 0 < \phi < 2\pi, 0 < z < 3.$
    \begin{equation}
        \nabla \cdot \hat{H} = \frac{1}{\rho}\frac{\partial}{\partial\rho}(4p^3)+2\frac{\partial}{\partial z}(z) = 122
    \end{equation}
    \begin{center}
        Therefore it verifies the theorem.
    \end{center}
    \newpage
    \item (10 Points) If $F = 2\rho za_p+3z\sin\phi a_\phi-4\rho\cos\phi a_z,$ Verify Stokes's theorem for the open surface defined by $z = 1, 0 < \rho < 2, 0 < \phi < 45^\circ$
    \begin{equation}
        3\int_0^2 (-\cos(45^\circ)-(-cos(0^\circ)))dS = \boxed{1.75358}
    \end{equation}
    \begin{center}
        Therefore this verifies stokes theorem
    \end{center}
    \newpage
    \item (10 Points) A point charge a the origin has a charge $-24nC.$ The medium surrounding the charge has relative permittivity equal to 2. What is the electric field intensity at( x = 1m, y = 2m, z =3m)? please give your answer in Cartesian coordinates
    \begin{align}
        E &= \frac{kQ}{r^2}\\
        &= \frac{2(9\cdot10^9)(-24nC)}{(\sqrt{1^2 + 2^2+ 3^2})^2}\\
        &= \frac{(18\cdot10^9)(-24nC)}{(\sqrt{14})^2}\\
        &= \frac{432}{14}\\
        &= \boxed{30.86}
    \end{align}
    \newpage
    \item (10 Points) point charges equal to 3 nC are located at (x,y,z) = (0,0, -0.5m) and (0, 0, +0.5) in free space.
    \begin{enumerate}
        \item  find the electric field intensity at (+1.5m, 0,0)
        \begin{align}
            E &= \frac{kQ}{r^2}\\
            &= \frac{(9\cdot10^9)(3nC)}{(\sqrt{1.5^2+.5^2})^2} + \frac{(9\cdot10^9)(3nC)}{(\sqrt{1.5^2+(-.5)^2})^2}\\
            &= \boxed{21.6}
        \end{align}
        \item What single point charge, located at (0,0,0), would result in the same electric field intensity?
        \begin{align}
            &21.6 = \frac{(9\cdot10^9)Q}{(\sqrt{1.5^2})^2}\\
            &\Rightarrow Q = \frac{(1.5^2)(21.6)}{9\cdot10^9}\\
            &\Rightarrow Q = \boxed{5nC}
        \end{align}
    \end{enumerate}
    \newpage
    \item (10 Points) Three infinite flat sheets of charge exist in a medium with permittivity equal to twice that of free space. The first sheet lies in the x = 0 plane and has constant charge density $+4nC/m^2$. The second sheet lies in the y = 0 plane and has constant charge density $+16nC/m^2$. The third sheet lies in the z = 0 plane and has constant charge density $+64nC/m^2$. What is the electric field intensity in the region $\{x > 0, y > 0, z > 0\}$? Please do not lead you answer in terms of physical constants, and instead reduce to values as much as possible.
    \begin{align}
        E &= \hat{x}\frac{\rho_{s,1}}{4\epsilon_0} + \hat{y}\frac{\rho_{s,2}}{4\epsilon_0} + \hat{z}\frac{\rho_{s,3}}{4\epsilon_0}\\
        &= \boxed{\hat{x}113V/m + \hat{y}452V/m + \hat{z}1807V/m}
    \end{align}
    \newpage
    \item (10 Points) An infinite line of charge having uniform density +8mC/m exist along the z-axis. In addition, a infinite sheet of charge having uniform charge density $+12mC/m^2$ lies in the z = 0 plane. If the permittivity of them medium is twice that of free space, then what is the electric field intensity in the region $z > 0$? Express your answer in cylindrical coordinates.
    \begin{equation}
        E = E_{line} + E_{sheet} = \hat{\rho}\frac{\rho_l}{2\pi\epsilon\rho}+\hat{z}\frac{\rho_s}{2\epsilon}
    \end{equation}
    \begin{equation}
        E = \boxed{\hat{\rho}\frac{71.9MV}{\rho} + \hat{z}(338.8\frac{MV}{m})}
    \end{equation}
    \newpage
    \item (10 Points) An infinite line of charge having a charge density $\rho_l$ exist along  z-axis. Find the electric potential difference $V_{21}$ between points at distances $\rho_1$ and $\rho_2$ along a radial extending from the line of charge. Leave $\rho_l$ and the permittivity of the medium $(\epsilon)$ as independent variables. Note: You do not need to derive an expression for the electric field due to the line charge; feel free to use an expression from a textbook for this.
    \begin{equation}
        E(\rho) = \hat{\rho}\frac{\rho_l}{2\pi\epsilon\rho}
    \end{equation}
    \begin{equation}
        V_{21} = -\int_{p1}^{p2}E\cdot d\hat{l} = -\int_{p1}^{p2} \hat{\rho}\frac{\rho_l}{2\pi\epsilon\rho}\cdot\hat{\rho}d\rho = \boxed{\frac{\rho_l}{2\pi\epsilon}\ln\frac{\rho_1}{\rho_2}}
    \end{equation}
    \newpage
    \item (5 Points) A parallel-plate capacitor has a capacitance of 20pF. If 3V is applied to this capacitor, what is the net charge in the capacitor? What is the charge on positively-charged plate?
    \begin{equation}
        Q = (20\cdot10^{-12})(3) = \boxed{60pC}
    \end{equation}
    \newpage
    \item (5 Points) A certain design requires the unintended capacitance contribute by a two-layer printed circuit board to be less than 3pF. the PCB has thickness 2mm and relative permittivity 3. How much area may be in common between the top layer and the bottom layer?
    \begin{align}
        &C < 3pF\\
        &\Rightarrow A < \frac{3\cdot10^{-12}2\cdot10^{-3}}{4\epsilon_o}\\
        &\Rightarrow \boxed{A < 135.59 mm^2}
    \end{align}
    \newpage
    \item  The capacitance of a particular coaxial cable is 30pF/m. This cable uses polyethylene as the spacer material, having a relative permittivity of 2.25. What is the capacitance of this cable if the spacer material is changed from polyethylene to air?
    \begin{equation}
        \boxed{C_{air} = \frac{30}{2.25}pF/m}
    \end{equation}
\end{enumerate}
\end{document}