% DOC SETTINGS ===================================
\documentclass{article}
\usepackage[utf8]{inputenc}
\usepackage{steinmetz}
\usepackage{mathtools}  
\usepackage{multicol}
\usepackage{circuitikz}
\usepackage{listings}
\usepackage{geometry}
\usepackage{fancyhdr}
\usepackage{amssymb}
\usepackage{pifont}
\usetikzlibrary{positioning, fit, calc}
\pagestyle{fancy}
\lhead{ECE2214 Homework 6}
\rhead{Kavin Thirukonda 2021}
\fancyheadoffset{0mm}
\geometry{
 a4paper,
 total={170mm,257mm},
 left=20mm,
 top=25mm,
 }
\mathtoolsset{showonlyrefs} 
\cfoot{}
% DOC SETTINGS ===================================
\begin{document}
\begin{enumerate}
    \item (20 Points) A thin wire in free space lies along the z-axis and carries a current of I = 3A flowing in the +z direction. In the same plane lies a rectangular loop, 1cm in the $\rho$ dimension and 10 cm in the z direction, with the closest side being 1cm from the wire. What is the magnetic flux through the loop?
    \begin{equation}
        B = \frac{\mu_oI}{2\pi rho}
    \end{equation}
    \begin{align}
        \Phi &= \int_{1cm}^{2cm} \int_{0cm}^{10cm}  \frac{\mu_oI}{2\pi r} d\rho dz\\
        &= \frac{\mu_o3}{2\pi \rho}\int_{1cm}^{2cm} 0.1dz\\
        &= \frac{\mu_o3}{2\pi \rho}(0.1\cdot0.2 - 0.1 \cdot0.01)\\
        &= \frac{\mu_o3}{2\pi \rho}(0.02 - 0.001)\\
        &= \frac{\mu_o3}{2\pi \rho}(0.019)
        &= \boxed{41.6 nWb}
    \end{align}
    \item A wire having circular cross-section of radius a = 5mm lies along the z-axis and has an internal magnetic field given by:
    \begin{equation}
        B = \hat{\phi}\mu_oJ_o\rho\text{ for } \rho < a
    \end{equation}
    \begin{equation}
        \int_{\phi = 0}^{2\pi}\hat{\phi}J_0a\cdot\hat{\phi}ad\phi = \boxed{20mA}
    \end{equation}
    where $J_o = 127.3 A/m^2$. Use Ampere's Law to find the total current carried by the wire
    \item A line current flows along the y-axis, in the -$\hat{y}$ direction. In what direction does the magnetic field point at (x, y, z) = (+1, +1, 0)m? Give your answer in Cartesian coordinates.
    \begin{center}
        Using the righthand rule it points in the $\boxed{+\hat{z}}$ direction at the given point
    \end{center}
    \item A cardboard tube is 10cm long and 5mm in diameter, with air inside ("air core"). 100 windings of insulated wire are wound onto the tube, and connected to a 2A current source, Also, 300 windings of insulated wire wound onto the same tube but in the opposite direction, and connected to a 4A current source. The two sets of windings are overlapping, but they do not short out because they are insulated. What is the magnitude of the magnetic flux density in the coil.
    \begin{equation}
        \left|\mu_0(\frac{N_1I_1}{l}-\frac{N_2I_2}{l})\right| = \boxed{12.6mT}
    \end{equation}
    \item What is the inductance of a solenoid which is 5 cm in length, 5 mm in diameter, and consists of 300 turns of wire, if the core of the solenoid is a ferromagnetic material having relative permeability equal to 200.
    \begin{equation}
        \Phi = \frac{LI}{2}
    \end{equation}
    \item Consider an inductor constructed from linear and time-invariant materials, and which has an inductance of 1 H. The magnetic field within the inductor is caused to double by changing the current. What does the inductance become?
    \begin{center}
        remains 1H
    \end{center}
\end{enumerate}
\end{document}/