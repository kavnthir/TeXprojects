% DOC SETTINGS ===================================
\documentclass{article}
\usepackage[utf8]{inputenc}
\usepackage{steinmetz}
\usepackage{mathtools}  
\usepackage{multicol}
\usepackage{circuitikz}
\usepackage{tikz}
\usepackage{listings}
\usepackage{geometry}
\usepackage{fancyhdr}
\usepackage{amsfonts}
\usepackage{media9}
\usepackage{parskip}
\usetikzlibrary{positioning, fit, calc}
\pagestyle{fancy}
\lhead{Grayscale Research}
\rhead{Kavin Thirukonda 2021}
\PassOptionsToPackage{hyphens}{url}\usepackage{hyperref}
\hypersetup{
    colorlinks=true,
    linkcolor=blue,
    filecolor=magenta,      
    urlcolor=cyan,
}
\fancyheadoffset{0mm}
 \geometry{
 a4paper,
 total={170mm,257mm},
 left=20mm,
 top=25mm,
 }
\usepackage{listings}
\usepackage[]{color}
\definecolor{codegreen}{rgb}{0,0.6,0}
\definecolor{codegray}{rgb}{0.5,0.5,0.5}
\definecolor{codepurple}{rgb}{0.58,0,0.82}
\definecolor{backcolour}{rgb}{0.95,0.95,0.92}
\mathtoolsset{showonlyrefs} 
\cfoot{}
% DOC SETTINGS ===================================
\begin{document}
\section*{Grayscale}
\section*{Stochastic Realistic Film Grain}
\begin{center}
    \url{https://hal.archives-ouvertes.fr/hal-01494123v2/file/film_grain_gaussian_texture.pdf}
    
    Paper describing the need and use of having a physically realistic film grain on images and movies, aswell as providing the math and statistics to implement them.
    
    \url{https://pdfs.semanticscholar.org/b791/566b0a38cc0269b1b7f9adbe5ba569aa4eb5.pdf}
    
    Similar paper covering the same general topics, but seems to go in to more detail and explain certain concepts better
\end{center}
\end{document} 