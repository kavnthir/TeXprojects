% DOC SETTINGS ===================================
\documentclass{article}
\usepackage[utf8]{inputenc}
\usepackage{fancyhdr}
\pagestyle{fancy}
\usepackage{listings}
\usepackage{geometry}
 \geometry{
 a4paper,
 total={170mm,257mm},
 left=20mm,
 top=25mm,
 }
\fancyheadoffset{0mm}
\lhead{STAT4714 Homework 11}
\rhead{Kavin Thirukonda 2021}
\usepackage{steinmetz}
\usepackage{mathtools}  
\usepackage{multicol}
\mathtoolsset{showonlyrefs} 
\cfoot{}
% DOC SETTINGS ===================================
\begin{document}
\section*{Textbook}
\subsection*{Problem 32}
A discrete random variable has moment generating function 
\begin{equation}
    m_X(t) = e^{2(e^t-1)}
\end{equation}
\subsubsection*{a)}
Find E[X]
\begin{align}
    E[X] &= m^{'}_X(t)\bigg|_{t=0}\\
     &= \frac{d}{dt}(e^{2(e^t-1)})\bigg|_{t=0}\\
     &= 2e^{2\left(e^s-1\right)+s}\bigg|_{t=0}\\
     &= 2e^{2(1-1)}\\
     &= \boxed{2}
\end{align}
\subsubsection*{b)}
Find E[$X^2$]
\begin{align}
    E[X^2] &= m^{''}_X(t)\bigg|_{t=0}\\
     &= \frac{d^2}{dt^2}(e^{2(e^t-1)})\bigg|_{t=0}\\
     &= 2e^{2e^s-2+s}\left(2e^s+1\right)\bigg|_{t=0}\\
     &= 2e^{2-2}\left(2+1\right)\\
     &= \boxed{6}
\end{align}
\subsubsection*{c)}
Find $\sigma^2$ and $\sigma$
\begin{equation}
    \sigma^2 = 6 - 2^2 = \boxed{2}
\end{equation}
\begin{equation}
    \sigma = \sqrt{\sigma^2} = \boxed{\sqrt{2}}
\end{equation}
\newpage
\subsection*{Problem 37}
In each part the moment generating function for a random variable X is given. Identify the family to which the random variable belongs, and give the numerical values of pertinent distribution parameters.
\subsubsection*{a)}
\begin{equation}
    m_X(t) = e^{2t+\frac{9t^2}{2}}
\end{equation}
\begin{center}
    Normal distribution,
    $\mu = 2,\text{ }\sigma^2 = 9$
\end{center}
\subsubsection*{b)}
\begin{equation}
    m_X(t) = e^{8t^2}
\end{equation}
\begin{center}
    normal distribution,
    $\mu = 0,\text{ }\sigma^2 = 16$
\end{center}
\subsubsection*{c)}
\begin{equation}
    m_X(t) = \frac{.25e^t}{1-.75e^t}  
\end{equation}
\begin{center}
    geometric distribution,
    $p = 0.25$
\end{center}
\subsubsection*{d)}
\begin{equation}
    m_X(t) = (.5+.5e^t)^5    
\end{equation}
\begin{center}
    binomial distribution,
    $n = 5,\text{ }p = 0.5$
\end{center}
\subsubsection*{e)}
\begin{equation}
    m_X(t) = e^{6(e^t-1)}
\end{equation}
\begin{center}
    poisson distribution,
    $k=6$
\end{center}
\subsubsection*{f)}
\begin{equation}
    m_X(t) = (1-3t)^{-5}
\end{equation}
\begin{center}
    gamma distribution,
    $\alpha = 5,\text{ }\beta = 3$
\end{center}
\subsubsection*{g)}
\begin{equation}
    m_X(t) = (1-2t)^{-8}
\end{equation}
\begin{center}
    chi-squared distribution,
    $\gamma = 16$
\end{center}
\subsubsection*{h)}
\begin{equation}
    m_X(t) = (1-.5t)^{-1}
\end{equation}
\begin{center}
    exponential distribution,
    $\beta = 0.5$
\end{center}
\newpage
\section*{Canvas}
\subsection*{Problem 21}
In a pressure test, the water pressure Pin kilograms per square centimeter through a certain valve will be steadily increased until the valve breaks, at some uncertain pressure P = p. Theory suggests that the moment generating function of the break pressure will be 
\begin{equation}
    m_p(s) = \frac{0.4}{1 -1 0s} + \frac{0.6}{1-15}
\end{equation}
\subsubsection*{a)}
Find the expected value of the break pressure      
\begin{align}
    E[X] &= m^{'}_p(s)\bigg|_{s=0} \\
    &= \frac{d}{ds}\left(\frac{0.4}{1 -1 0s} + \frac{0.6}{1-15}\right)\bigg|_{s=0} \\
    &= \frac{4}{\left(1-10s\right)^2}+\frac{9}{\left(1-15s\right)^2}\bigg|_{s=0}\\
    &= \frac{4}{1^2}+\frac{9}{1^2}\\
    &= \boxed {13}
\end{align}
\subsubsection*{b)}
Find the variance of the break pressure   
\begin{align}
    E[X^2] &= m^{''}_p(s)\bigg|_{s=0} \\
    &= \frac{d^2}{ds^2}\left(\frac{0.4}{1 -1 0s} + \frac{0.6}{1-15}\right)\bigg|_{s=0} \\
    &= -\frac{50\left(12s-1\right)\left(900s^2-150s+7\right)}{\left(10s-1\right)^3\left(15s-1\right)^3}\bigg|_{s=0}\\
    &= -\frac{50\left(-1\right)\left(7\right)}{\left(-1\right)^3\left(-1\right)^3}\\
    &= 350
\end{align}
\begin{equation}
    \sigma^2 = 350 - 13^2 = \boxed{181}
\end{equation}
\newpage
\subsection*{Problem 27}
If you graph some Gamma$(\alpha, \beta)$ densities, you will notice that in some cases as \textit{t} increases they increase from zero, reach a maximum, and then decrease. Use elementary calculus to find, for $\alpha > 1$, the value of \textit{t} for which the density reaches a maximum. (This value of \textit{t} is called the \textbf{mode} of the distribution.)
\begin{equation}
    f(t) = \underbrace{\frac{\beta^\alpha}{\Gamma(\alpha)}}_{\text{not t reliant}}x^{\alpha-1}e^{-\beta t}
\end{equation}
\begin{align}
    & f^{'}(t) =0\\
    & \frac{d}{dt}(t^{\alpha-1}e^{-\beta t}) =0\\
    &\Rightarrow e^{-t\beta}t^{\alpha-2}\left(\alpha-1\right)-e^{-t\beta}\beta t^{\alpha-1} = 0\\
    &\Rightarrow e^{-t\beta}[t^{\alpha-2}\left(\alpha-1\right)-\beta t^{\alpha-1}] = 0\\
    &\Rightarrow \underbrace{t^{\alpha-2}}_{0\text{ when }t = 0 }e^{-t\beta}\underbrace{[\left(\alpha-1\right)-\beta t]}_{0\text{ when }t = \frac{\alpha - 1}{\beta}} = 0
\end{align}
\begin{center}
    Therefore the mode is reached when $t = \frac{\alpha - 1}{\beta}$
\end{center}
\subsection*{Problem 28}
As in Exercise 27, notice that for Weibull(a, b) densities sometimes increase as \textit{t} increases from zero, reach a maximum, and then decrease. For $a > 1$, find the value of \textit{t} that is the mode of the distribution. (We here use our notation for Weibull(a, b) densities
\begin{equation}
    H(t)  = \left(\frac{t}{b}\right)^a
\end{equation}
\begin{center}
    Using the same process as the last question we get that
\end{center}
\begin{equation}
    t = b(1-\frac{1}{a})^{\frac{1}{a}}
\end{equation}
\end{document} 